\documentclass[12pt,a4paper]{article}
\usepackage[utf8]{inputenc}
\usepackage[T1]{fontenc}
\usepackage{amsmath}
\usepackage{amssymb}
\usepackage{graphicx}
\usepackage[left=2.50cm, right=2.50cm, top=2.50cm, bottom=2.50cm]{geometry}

\usepackage{helvet} %ARIAL
\renewcommand{\familydefault}{\sfdefault}

\usepackage{fancyhdr}
\usepackage{tabularx}
\usepackage[table]{xcolor}
\usepackage{cancel}
 \usepackage{lscape}
 \usepackage{mathtools}
 \usepackage{mathrsfs}

\begin{document}
	\pagestyle{fancy}
	\renewcommand{\headrulewidth}{0pt}
	\fancyhead{}
	\fancyfoot{}
	
	\begin{landscape}
%		\begin{flushleft}	
%			CUERPO EA\\\vspace{0.2cm}
%			$\sum F_{x}$\\\vspace{0.2cm}
%			$E_{x}+A_{x}=2\cdot masa_{EA} \cdot a_{X}$\\\vspace{0.2cm}
%			$\sum F_{y}$\\\vspace{0.2cm}
%			$	E_{y}+A_{y}-P_{eso~AE}=2\cdot masa_{EA
%			} \cdot a_{y}$ \\\vspace{0.2cm}
%			$\sum M_{E }=\sum M_{E}efectivos$\\\vspace{0.2cm}
%			$-100 Cos(45) \cdot P_{eso~AE}-200Sen(45)\cdot A_{x}+200 Cos(45)\cdot A_{y} = 2I_{nercia~respecto~E} \cdot \alpha$\\\vspace{0.8cm}
%			
%			CUERPO OB\\\vspace{0.2cm}
%			$\sum F_{x}$\\\vspace{0.2cm}
%			$O_{x}+B_{x}=2\cdot masa_{OB} \cdot a_{X}$\\\vspace{0.2cm}
%			$\sum F_{y}$\\\vspace{0.2cm}
%			$	O_{y}+B_{y}-P_{eso~OB}=2\cdot masa_{OB
%			} \cdot a_{y}$ \\\vspace{0.2cm}
%			$\sum M_{o }=\sum M_{o}efectivos$\\\vspace{0.2cm}
%			$-Mo-100 Cos(30) \cdot P_{eso~OB}-200Sen(30)\cdot B_{x}+200 Cos(45)\cdot B_{y} = 2I_{nercia~respecto~O} \cdot \alpha$\\\vspace{0.8cm}
%			
%			CUERPO ABP\\\vspace{0.2cm}
%			$\sum F_{x}$\\\vspace{0.2cm}
%			$-A_{x}-B_{x}=masa_{ABP} \cdot a_{X}$\\\vspace{0.2cm}
%			$\sum F_{y}$\\\vspace{0.2cm}
%			$	-A_{y}-B_{y}-P_{eso~ABP}= masa_{ABP
%			} \cdot a_{y}$ \\\vspace{0.2cm}
%			$\sum M_{B }=\sum M_{B}efectivos$\\\vspace{0.2cm}
%			$-150.07\cdot P_{eso~ABP} +150 Cos(28) \cdot A_{y} + 150Sen(28)\cdot A_{x} = I_{nercia~respecto~B} \cdot \alpha + masa_{ABP} \cdot a_{X} \cdot 3.97 + masa_{ABP} \cdot a_{Y} \cdot 150.07$\\\vspace{0.8cm}
%			
%			\newpage
%			2 	CUERPO EA\\\vspace{0.2cm}
%			$\sum F_{x}$\\\vspace{0.2cm}
%			$E_{x}+A_{x}=2\cdot masa_{EA} \cdot a_{X}$\\\vspace{0.2cm}
%			$\sum F_{y}$\\\vspace{0.2cm}
%			$	E_{y}+A_{y}-P_{eso~AE}=2\cdot masa_{EA
%			} \cdot a_{y}$ \\\vspace{0.2cm}
%			$\sum M_{E }=\sum M_{E}efectivos$\\\vspace{0.2cm}
%			$-100 Cos(20) \cdot P_{eso~AE}-200Sen(20)\cdot A_{x}+200 Cos(20)\cdot A_{y} = 2I_{nercia~respecto~E} \cdot \alpha$\\\vspace{0.8cm}
%			
%			CUERPO OB\\\vspace{0.2cm}
%			$\sum F_{x}$\\\vspace{0.2cm}
%			$O_{x}+B_{x}=2\cdot masa_{OB} \cdot a_{X}$\\\vspace{0.2cm}
%			$\sum F_{y}$\\\vspace{0.2cm}
%			$	O_{y}+B_{y}-P_{eso~OB}=2\cdot masa_{OB
%			} \cdot a_{y}$ \\\vspace{0.2cm}
%			$\sum M_{o }=\sum M_{o}efectivos$\\\vspace{0.2cm}
%			$-Mo-100 Cos(30) \cdot P_{eso~OB}+200\cdot B_{y} = 2I_{nercia~respecto~O} \cdot \alpha$\\\vspace{0.8cm}
%			
%			CUERPO ABP\\\vspace{0.2cm}
%			$\sum F_{x}$\\\vspace{0.2cm}
%			$-A_{x}-B_{x}=masa_{ABP} \cdot a_{X}$\\\vspace{0.2cm}
%			$\sum F_{y}$\\\vspace{0.2cm}
%			$	-A_{y}-B_{y}-P_{eso~ABP}= masa_{ABP
%			} \cdot a_{y}$ \\\vspace{0.2cm}
%			$\sum M_{B }=\sum M_{B}efectivos$\\\vspace{0.2cm}
%			$-150.07\cdot P_{eso~ABP} +150 Cos(41) \cdot A_{y} + 150Sen(41)\cdot A_{x} = I_{nercia~respecto~B} \cdot \alpha + masa_{ABP} \cdot a_{X} \cdot 3.97 + masa_{ABP} \cdot a_{Y} \cdot 156.07$\\\vspace{0.8cm}
%			
%						\newpage
%			3	CUERPO EA\\\vspace{0.2cm}
%	$\sum F_{x}$\\\vspace{0.2cm}
%	$E_{x}+A_{x}=2\cdot masa_{EA} \cdot a_{X}$\\\vspace{0.2cm}
%	$\sum F_{y}$\\\vspace{0.2cm}
%	$	E_{y}+A_{y}-P_{eso~AE}=2\cdot masa_{EA
%	} \cdot a_{y}$ \\\vspace{0.2cm}
%	$\sum M_{E }=\sum M_{E}efectivos$\\\vspace{0.2cm}
%	$-100 Cos(-30) \cdot P_{eso~AE}-200Sen(-30)\cdot A_{x}+200 Cos(-30)\cdot A_{y} = 2I_{nercia~respecto~E} \cdot \alpha$\\\vspace{0.8cm}
%	
%	CUERPO OB\\\vspace{0.2cm}
%	$\sum F_{x}$\\\vspace{0.2cm}
%	$O_{x}+B_{x}=2\cdot masa_{OB} \cdot a_{X}$\\\vspace{0.2cm}
%	$\sum F_{y}$\\\vspace{0.2cm}
%	$	O_{y}+B_{y}-P_{eso~OB}=2\cdot masa_{OB
%	} \cdot a_{y}$ \\\vspace{0.2cm}
%	$\sum M_{o }=\sum M_{o}efectivos$\\\vspace{0.2cm}
%	$-Mo+200\cdot B_{x} = 2I_{nercia~respecto~O} \cdot \alpha$\\\vspace{0.8cm}
%	
%	CUERPO ABP\\\vspace{0.2cm}
%	$\sum F_{x}$\\\vspace{0.2cm}
%	$-A_{x}-B_{x}=masa_{ABP} \cdot a_{X}$\\\vspace{0.2cm}
%	$\sum F_{y}$\\\vspace{0.2cm}
%	$	-A_{y}-B_{y}-P_{eso~ABP}= masa_{ABP
%	} \cdot a_{y}$ \\\vspace{0.2cm}
%	$\sum M_{B }=\sum M_{B}efectivos$\\\vspace{0.2cm}
%	$-3.97\cdot P_{eso~ABP} +150 Sen(29) \cdot A_{y} +150 Cos(29)\cdot A_{x} = I_{nercia~respecto~B} \cdot \alpha + masa_{ABP} \cdot a_{X}\cdot 150.17 + masa_{ABP} \cdot a_{Y}\cdot 3.97$\\\vspace{0.8cm}
%		\end{flushleft}
%	
%%		\begin{align}
%%			\sum F_{x}&\\
%%			E_{x}+A_{x}=&2\cdot masa_{EA} \cdot a_{X}\\
%%			\sum F_{y}&\\
%%				E_{y}+A_{y}-P_{eso~AE}=&2\cdot masa_{EA
%%			} \cdot a_{y}\\
%%			\sum M_{E }=&\sum M_{E}efectivos\\
%%			-100 Cos(45) \cdot P_{eso~AE}-200Sen(45)\cdot A_{x}+200 Cos(45)\cdot A_{y} =& 2I_{nercia~respecto~E} \cdot \alpha			
%%%			CUERPO OB\\\vspace{0.2cm}
%%%			\sum F_{x}$\\\vspace{0.2cm}
%%%			O_{x}+B_{x}=2\cdot masa_{OB} \cdot a_{X}$\\\vspace{0.2cm}
%%%			\sum F_{y}$\\\vspace{0.2cm}
%%%				O_{y}+B_{y}-P_{eso~OB}=2\cdot masa_{OB
%%%			} \cdot a_{y}$ \\\vspace{0.2cm}
%%%			\sum M_{o }=\sum M_{o}efectivos$\\\vspace{0.2cm}
%%%			-Mo-100 Cos(30) \cdot P_{eso~OB}-200Sen(30)\cdot B_{x}+200 Cos(45)\cdot B_{y} = 2I_{nercia~respecto~O} \cdot \alpha$\\\vspace{0.8cm}
%%%			
%%%			CUERPO ABP\\\vspace{0.2cm}
%%%			\sum F_{x}$\\\vspace{0.2cm}
%%%			-A_{x}-B_{x}=masa_{ABP} \cdot a_{X}$\\\vspace{0.2cm}
%%%			\sum F_{y}$\\\vspace{0.2cm}
%%%				-A_{y}-B_{y}-P_{eso~ABP}= masa_{ABP
%%%			} \cdot a_{y}$ \\\vspace{0.2cm}
%%%			\sum M_{B }=\sum M_{B}efectivos$\\\vspace{0.2cm}
%%%			-150.07\cdot P_{eso~ABP} +150 Cos(28) \cdot A_{y} + 150Sen(28)\cdot A_{x} = I_{nercia~respecto~B} \cdot \alpha + masa_{ABP} \cdot a_{X} \cdot 3.97 + masa_{ABP} \cdot a_{Y} \cdot 150.07$\\\vspace{0.8cm}
%%		\end{align}
%
%
%%	\begin{align}
%%
%%	E_{x}+A_{x}=&2\cdot masa_{EA} \cdot a_{X}\\
%%
%%	E_{y}+A_{y}-P_{eso~AE}=&2\cdot masa_{EA
%%	} \cdot a_{y}\\
%%
%%	-100 Cos(45) \cdot P_{eso~AE}-200Sen(45)\cdot A_{x}+200 Cos(45)\cdot A_{y} =& 2I_{nercia~respecto~E} \cdot \alpha			
%%	%			CUERPO OB\\\vspace{0.2cm}
%%	%			\sum F_{x}$\\\vspace{0.2cm}
%%	%			O_{x}+B_{x}=2\cdot masa_{OB} \cdot a_{X}$\\\vspace{0.2cm}
%%	%			\sum F_{y}$\\\vspace{0.2cm}
%%	%				O_{y}+B_{y}-P_{eso~OB}=2\cdot masa_{OB
%%		%			} \cdot a_{y}$ \\\vspace{0.2cm}
%%	%			\sum M_{o }=\sum M_{o}efectivos$\\\vspace{0.2cm}
%%	%			-Mo-100 Cos(30) \cdot P_{eso~OB}-200Sen(30)\cdot B_{x}+200 Cos(45)\cdot B_{y} = 2I_{nercia~respecto~O} \cdot \alpha$\\\vspace{0.8cm}
%%	%			
%%	%			CUERPO ABP\\\vspace{0.2cm}
%%	%			\sum F_{x}$\\\vspace{0.2cm}
%%	%			-A_{x}-B_{x}=masa_{ABP} \cdot a_{X}$\\\vspace{0.2cm}
%%	%			\sum F_{y}$\\\vspace{0.2cm}
%%	%				-A_{y}-B_{y}-P_{eso~ABP}= masa_{ABP
%%		%			} \cdot a_{y}$ \\\vspace{0.2cm}
%%	%			\sum M_{B }=\sum M_{B}efectivos$\\\vspace{0.2cm}
%%	%			-150.07\cdot P_{eso~ABP} +150 Cos(28) \cdot A_{y} + 150Sen(28)\cdot A_{x} = I_{nercia~respecto~B} \cdot \alpha + masa_{ABP} \cdot a_{X} \cdot 3.97 + masa_{ABP} \cdot a_{Y} \cdot 150.07$\\\vspace{0.8cm}
%%\end{align}
%	
%	
%	1
%	\begin{center}
%		\begin{equation*}
%			\begin{pmatrix*}[r]
%				E_{x}+A_{x}\\
%				E_{y}+A_{y}-P_{eso~AE}\\
%				-100 Cos(45) \cdot P_{eso~AE}-200Sen(45)\cdot A_{x}+200 Cos(45)\cdot A_{y}\\
%				O_{x}+B_{x}\\
%				O_{y}+B_{y}-P_{eso~OB}\\
%				-Mo-100 Cos(30) \cdot P_{eso~OB}-200Sen(30)\cdot B_{x}+200 Cos(45)\cdot B_{y}\\
%				-A_{x}-B_{x}\\
%				-A_{y}-B_{y}-P_{eso~ABP}\\
%				-150.07\cdot P_{eso~ABP} +150 Cos(28) \cdot A_{y} + 150Sen(28)\cdot A_{x}
%			\end{pmatrix*} 
% 				=
%			\begin{pmatrix*}[l]
%				2\cdot masa_{EA} \cdot a_{X}\\
%				2\cdot masa_{EA} \cdot a_{y}\\				
%				2I_{nercia~respecto~E} \cdot \alpha\\
%				2\cdot masa_{OB} \cdot a_{X}\\
%				2\cdot masa_{OB} \cdot a_{y}\\
%				2I_{nercia~respecto~O} \cdot \alpha\\
%				masa_{ABP} \cdot a_{X}\\
%				masa_{ABP} \cdot a_{y}\\
%				I_{nercia~respecto~B} \cdot \alpha + masa_{ABP} \cdot a_{X} \cdot 3.97 + masa_{ABP} \cdot a_{Y} \cdot 150.07
%			\end{pmatrix*} 
%		\end{equation*}
%	\end{center}
%2
%	\begin{center}
%	\begin{equation*}
%		\begin{pmatrix*}[r]
%			E_{x}+A_{x}\\
%			E_{y}+A_{y}-P_{eso~AE}\\
%			-100 Cos(20) \cdot P_{eso~AE}-200Sen(20)\cdot A_{x}+200 Cos(20)\cdot A_{y}\\ 
%			O_{x}+B_{x}\\
%			O_{y}+B_{y}-P_{eso~OB}\\
%			-Mo-100 Cos(30) \cdot P_{eso~OB}+200\cdot B_{y}\\
%			-A_{x}-B_{x}\\
%			-A_{y}-B_{y}-P_{eso~ABP}\\
%			-150.07\cdot P_{eso~ABP} +150 Cos(41) \cdot A_{y} + 150Sen(41)\cdot A_{x}
%		\end{pmatrix*} 
%		=
%		\begin{pmatrix*}[l]
%			2\cdot masa_{EA} \cdot a_{X}\\
%			2\cdot masa_{EA} \cdot a_{y}\\				
%			2I_{nercia~respecto~E} \cdot \alpha\\
%			2\cdot masa_{OB} \cdot a_{X}\\
%			2\cdot masa_{OB} \cdot a_{y}\\
%			2I_{nercia~respecto~O} \cdot \alpha\\
%			masa_{ABP} \cdot a_{X}\\
%			masa_{ABP} \cdot a_{y}\\
%			I_{nercia~respecto~B} \cdot \alpha + masa_{ABP} \cdot a_{X} \cdot 3.97 + masa_{ABP} \cdot a_{Y} \cdot 156.07
%		\end{pmatrix*} 
%	\end{equation*}
%\end{center}
%3
% 	\begin{center}
% 	\begin{equation*}
% 		\begin{pmatrix*}[r]
% 			E_{x}+A_{x}\\
% 			E_{y}+A_{y}-P_{eso~AE}\\
% 			-100 Cos(-30) \cdot P_{eso~AE}-200Sen(-30)\cdot A_{x}+200 Cos(-30)\cdot A_{y} \\ 
% 			O_{x}+B_{x}\\
% 			O_{y}+B_{y}-P_{eso~OB}\\
% 			-Mo+200\cdot B_{x}\\
% 			-A_{x}-B_{x}\\
% 			-A_{y}-B_{y}-P_{eso~ABP}\\
% 			-3.97\cdot P_{eso~ABP} +150 Sen(29) \cdot A_{y} +150 Cos(29)\cdot A_{x}
% 		\end{pmatrix*}   =
% 		\begin{pmatrix*}[l]
% 			2\cdot masa_{EA} \cdot a_{X}\\
% 			2\cdot masa_{EA} \cdot a_{y}\\				
% 			2I_{nercia~respecto~E} \cdot \alpha\\
% 			2\cdot masa_{OB} \cdot a_{X}\\
% 			2\cdot masa_{OB} \cdot a_{y}\\
% 			2I_{nercia~respecto~O} \cdot \alpha\\
% 			masa_{ABP} \cdot a_{X}\\
% 			masa_{ABP} \cdot a_{y}\\
% 			I_{nercia~respecto~B} \cdot \alpha + masa_{ABP} \cdot a_{X}\cdot 150.17 + masa_{ABP} \cdot a_{Y}\cdot 3.97
% 		\end{pmatrix*} 
% 	\end{equation*}
% \end{center}
Acá probando el branch h a ver que onda.
$T = \frac{1}{2}~m~v^{2}+\frac{1}{2}~I~\omega^{2}$\\\vspace{0.8cm}

$T_{sistema}=~2~T_{\text{rueda dentada chica}}+~T_{\text{rueda dentada grande}}+~T_{\text{rueda dentada tensora}}~+T_{\text{plataforma}}+~T_{\text{motor}}$
\end{landscape}
%
%\newpage
%
%
%
%$V_{s}= i_{t} ~ R~ + ~L  \dfrac{di}{dt}~+~$
%	
%\begin{tabular}{r l}\vspace{.2cm}	
%	Raíces & $s_{1-2}=-\alpha\pm \sqrt{\alpha^{2}-\omega_{0}^{2}}$\\
%	Frecuencia angular & $\omega = 2 \pi f = 2 \pi \dfrac{1}{\tau} $\vspace{.2cm}\\ 
%	
%	\multicolumn{2}{l}{Circuito RLC \textsl{en serie} (\textbf{LTK})}\vspace{.4cm} \\
%	\multicolumn{2}{c}{$iR+L\dfrac{di}{dt}+\dfrac{1}{C}\displaystyle\int_{-\infty}^{t}idt=0 $} \vspace{.4cm} \\ 
%	\multicolumn{2}{c}{$\dfrac{di}{dt}R+L\dfrac{d^2i}{dt^2}+\dfrac{i}{C}=0 $} \vspace{.8cm} \\ 
%
%	
%
%	
%	
%	\multicolumn{2}{l}{Circuito RLC \textsl{en paralelo} (\textbf{LCK})}\vspace{.4cm} \\
%	
%	\multicolumn{2}{c}{$\dfrac{v}{R}+\dfrac{1}{L}\displaystyle\int_{-\infty}^{t}v dt+C\dfrac{dv}{dt}=0 $} \vspace{.4cm} \\ 
%	
%	\multicolumn{2}{c}{$\dfrac{1}{R}\dfrac{dv}{dt}+\dfrac{1}{L} v +C\dfrac{d^2 v}{dt^2}=0 $} \vspace{.8cm} \\ 
%	
%
%
%
%\end{tabular}\\
%
%
% \begin{align*}
% 	  R.\mathcal{L}\{i'(t)\} +L.\displaystyle\mathcal{L}\{i''(t)\}+\dfrac{1}{C}\mathcal{L}\{i(t)\}=&0\\
% 	  R(s.F(s)-f(0)) +L	(s^{2}.F(s)-s.f(0)-f'(0))+\dfrac{1}{C}.F(s)=&0\\ 
% 	  F(s).\left(L.s^{2}+R.s+\dfrac{1}{C}\right)-f'(0).L-f(0).(R+Ls)=&0\\ 
% 	F(s)=&\dfrac{f'(0).L+f(0).(R+Ls)}{\left(L.s^{2}+R.s+\dfrac{1}{C}\right)}\\	
% 	F(s)=&\dfrac{A.s+B}{\left(D.s^{2}+E.s+G\right)}\\
% 	F(s)=&\dfrac{A.s}{\left(D.s^{2}+E.s+G\right)}+\dfrac{B}{\left(D.s^{2}+E.s+G\right)}\\
% \end{align*}

% \begin{align*}
% 	\dfrac{1}{R}\mathcal{L}\{v'(t)\}+\dfrac{1}{L} \mathcal{L}\{v(t)\} +C\mathcal{L}\{v''(t)\}=&0\\
% 	\dfrac{1}{R}(s.F(s)-f(0))+\dfrac{1}{L} .F(s)+C	(s^{2}.F(s)-s.f(0)-f'(0))=&0\\
% 	%
% 	F(s).\left(\dfrac{s}{R}+\dfrac{1}{L}+C.s^{2}\right)-\dfrac{f(0)}{R}-s.C.f(0)-f'(0).C=&0\\ 
% 	F(s)=&\dfrac{f(0)\left(\dfrac{1}{R}+s.C\right)+f'(0).C}{\left(\dfrac{s}{R}+\dfrac{1}{L}+C.s^{2}\right)}\\	
% 	F(s)=&\dfrac{A.s+B}{\left(D.s^{2}+E.s+G\right)}\\
% 	F(s)=&\dfrac{A.s}{\left(D.s^{2}+E.s+G\right)}+\dfrac{B}{\left(D.s^{2}+E.s+G\right)}\\
% \end{align*}
%
%$ V_{completa}=V_{s}+f(t)$
%
%
%$ I_{completa}=I_{forzada}+f(t)$
% EJ8\\
% a- $233/9$\\
% b- $17/4$\\
% c- $383/90$\\
% d- $2743/900$\\
% e- $151/990$\\
% f- $123031/99900$\\
% EJ9\\
% a- $29/3$\\
% b- $83/12$\\
% c- $ -31/15$\\
% d- $1/10$\\
% e- $2/9$\\
% f- $54/35$\\
% g- $-1$\\
% h- $ 81/256$\\
% i- $-4 $\\
% j- $ -36$\\
% k- $5/3$\\
% l- $53/9$\\
% m- $-215/24$\\
% n- $15$\\
% o- $323/9$\\
% EJ10\\
% a- 1\\
% b- $7/5$\\
% c- $2/3$\\
% d- $10/27$\\
% e- $191/13$\\

% EJ11\\
% a- $2a$\\
% b- $\dfrac{-5}{x^{2}-x} $\\
% c- $\dfrac{6xy^{2}+20x^{2}y-5xy}{10x^{2}y^{2}} $\\
% d- $\dfrac{a^{2}+b^{2}}{a^{2}-b^{2}} $\\\

% EJ12\\
% a- Emprendió el viaje con 50lts\\
% b- $\dfrac{3}{7}$\\
% c- 45\\
% d- ¿?\\
% EJ13\\
% a- n=-7\\
% b- n=16\\


% EJ14\\
% a- exponentes 3 y -7\\
% b- exponentes -5 y -2\\
% EJ15\\
% a-	$7\sqrt{2}$ \\
% b- $3\sqrt{2}-10\sqrt{5}$\\
% c- $-\sqrt{5}$\\
% d- $7.\sqrt[3]{2}$\\
% e-$ \sqrt[6]{2^{11}}$\\
% f- $\sqrt{3}.(9-2.\sqrt{2})$\\
% g-$1-\sqrt{2}$\\
% h- $ 4$\\
% i- $13/3 $\\
% EJ16\\
% a-	$\sqrt{2}+ 1$ \\
% b- $\sqrt{5} $ \\
% c- $ 2\sqrt{x}$ \\
% d- $\dfrac{2}{3}\sqrt{2} $ \\
% e-$\dfrac{1}{5}\sqrt{10} $ \\
% f- $ \dfrac{5+15\sqrt{2}}{17}$ \\
% EJ17\\
% perímetro=$2+6\sqrt{2}$ y área= $\sqrt{8}+4$\\
% EJ18\\
% a-	$4\times10^{2}$\\
% b-	$1/10$\\
% c-	$1/20$\\
% EJ19\\
% a-	$Log_{4}(x)+log_{4}(z)$\\
% b-	$Log(y)-Log(x)$\\
% c-	$5log(y)+2Log(w)-4log(x)-3log(z)$\\
% d-	$¼(7Ln(x)-5Ln(y)-Ln(z))$\\
% EJ20 ¿?\\
% EJ21\\
% a-	$Ln(6x)$\\
% b-	$Ln(5/x)$\\
% c-	$Log3(m.x^{2})$\\
% d-	$Log5(25x)$\\
% e-	$Log2(\dfrac{(r.x^{2/3})}{y})$\\
% f-	$Log2(\dfrac{\sqrt{y}}{x.z})$\\


\end{document}


